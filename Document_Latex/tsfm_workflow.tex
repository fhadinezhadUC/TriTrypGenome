\documentclass[table,
12pt, % Main document font size
a4paper, % Paper type, use 'letterpaper' for US Letter paper
oneside, % One page layout (no page indentation)
%twoside, % Two page layout (page indentation for binding and different headers)
headinclude,footinclude, % Extra spacing for the header and footer
BCOR5mm, % Binding correction
]{scrartcl}
%%%%%%%%%%%%%%%%%%%%%%%%%%%%%%%%%%%%%%%%%
% Arsclassica Article
% Structure Specification File
%
% This file has been downloaded from:
% http://www.LaTeXTemplates.com
%
% Original author:
% Lorenzo Pantieri (http://www.lorenzopantieri.net) with extensive modifications by:
% Vel (vel@latextemplates.com)
%
% License:
% CC BY-NC-SA 3.0 (http://creativecommons.org/licenses/by-nc-sa/3.0/)
%
%%%%%%%%%%%%%%%%%%%%%%%%%%%%%%%%%%%%%%%%%

%----------------------------------------------------------------------------------------
%	REQUIRED PACKAGES
%----------------------------------------------------------------------------------------

\usepackage[
nochapters, % Turn off chapters since this is an article        
beramono, % Use the Bera Mono font for monospaced text (\texttt)
eulermath,% Use the Euler font for mathematics
pdfspacing, % Makes use of pdftex’ letter spacing capabilities via the microtype package
dottedtoc % Dotted lines leading to the page numbers in the table of contents
]{classicthesis} % The layout is based on the Classic Thesis style

\usepackage{arsclassica} % Modifies the Classic Thesis package

\usepackage[T1]{fontenc} % Use 8-bit encoding that has 256 glyphs

\usepackage[utf8]{inputenc} % Required for including letters with accents

\usepackage{graphicx} % Required for including images
\graphicspath{{Figures/}} % Set the default folder for images

\usepackage{enumitem} % Required for manipulating the whitespace between and within lists

\usepackage{lipsum} % Used for inserting dummy 'Lorem ipsum' text into the template

\usepackage{subfig} % Required for creating figures with multiple parts (subfigures)

\usepackage{amsmath,amssymb,amsthm} % For including math equations, theorems, symbols, etc

\usepackage{varioref} % More descriptive referencing

%----------------------------------------------------------------------------------------
%	THEOREM STYLES
%---------------------------------------------------------------------------------------

\theoremstyle{definition} % Define theorem styles here based on the definition style (used for definitions and examples)
\newtheorem{definition}{Definition}

\theoremstyle{plain} % Define theorem styles here based on the plain style (used for theorems, lemmas, propositions)
\newtheorem{theorem}{Theorem}

\theoremstyle{remark} % Define theorem styles here based on the remark style (used for remarks and notes)

%----------------------------------------------------------------------------------------
%	HYPERLINKS
%---------------------------------------------------------------------------------------

\hypersetup{
%draft, % Uncomment to remove all links (useful for printing in black and white)
colorlinks=true, breaklinks=true, bookmarks=true,bookmarksnumbered,
urlcolor=webbrown, linkcolor=RoyalBlue, citecolor=webgreen, % Link colors
pdftitle={}, % PDF title
pdfauthor={\textcopyright}, % PDF Author
pdfsubject={}, % PDF Subject
pdfkeywords={}, % PDF Keywords
pdfcreator={pdfLaTeX}, % PDF Creator
pdfproducer={LaTeX with hyperref and ClassicThesis} % PDF producer
}
\usepackage{xcolor}
\usepackage[a4paper, total={6.5in, 10in}]{geometry}
\usepackage[edges]{forest}
\usepackage{tikz-qtree}
\usepackage{tcolorbox}
%\usepackage[table, dvipsnames]{xcolor}
%\usetikzlibrary{shapes.geometric,arrows.meta}
\colorlet{shadecolor}{gray!10}
\usepackage{cite}
\usepackage{adjustbox}
\usepackage{booktabs}
\usepackage{float}
\hyphenation{Fortran hy-phen-ation} 
\captionsetup{font=footnotesize}
\usepackage{booktabs}
\usepackage{enumitem}
\usepackage{tabularx, makecell}%
\usepackage{tikz}
\usepackage{subfig}
\usepackage{graphicx}
\usepackage{cleveref}
\usetikzlibrary{shapes.geometric, arrows}
\renewcommand\theadfont{\normalsize\bfseries}
        \usepackage{etoolbox} %
        \AtBeginEnvironment{tabularx}{\setlist[enumerate, 1]{wide, leftmargin=*, itemsep=0pt, before=\vspace{-\dimexpr\baselineskip +2 \partopsep}, after=\vspace{-\baselineskip}}}
%----------------------------------------------------------------------------------------
%       TITLE AND AUTHOR(S)
%----------------------------------------------------------------------------------------
\title{\normalfont\spacedallcaps{}} % The article title
\begin{document}
%----------------------------------------------------------------------------------------
%       HEADERS
%----------------------------------------------------------------------------------------
\renewcommand{\sectionmark}[1]{\markright{\spacedlowsmallcaps{#1}}} % The header for all pages 
\lehead{\mbox{\llap{\small\thepage\kern1em\color{halfgray} \vline}\color{halfgray}\hspace{0.5em}\rightmark\hfil}} % The header style
\pagestyle{scrheadings} % Enable the headers specified in this block
\setcounter{tocdepth}{3} % Set the depth of the table of contents to show sections and subsections only
\newpage 
%Input: the foreground and background informations calculated by tsfm for each state. states are the set of alphabets used in input gene sequences which are {A,T,C,G,-}\\
\section{Creating Function Logos, ID logos, KLD logos and bubble plot table with tsfm}
\begin{itemize}
\item[1. ] After preparing your Alignment results in folder \textbf{X }(Leish\_paper\_first\_round ...) for both sites72 and sites 74, run the script \textbf{extractFinalInput.sh} inside folder \textbf{X}, to prepare the input for tsfm in two subfolders called:\\
LogotaxAlignmentInputs/sites74/ and LogotaxAlignmentInputs/sites72/ \\
\item[2. ] For generating the logos of site72 we only need the folder site72 which has 9 subfolders one for each class of TriTryp genomes, and Homo. Each of these folders include the fasta files for 21 classes of tRNA.\\
\item[3. ] Keep the file \textbf{tRNA\_L\_skel\_Leish\_sites72\_struct.txt} and the script \textbf{generateLogos.sh} inside folder site72.\\
\item[4. ] In script generateLogos.sh, change the variable \textbf{tsfmpath} to the tsfm.py on your laptop as the example. The last version of tsfm is provided in folder tsfm, you can provide the path to tsfm.py from this folder.\\
\item[5. ] Finally:\\
cd workflow/site72\\
chmod +x generateLogos.sh\\
./generateLogos.sh\\
This should create all the function logos at first, ID logos second and KLD logos along with the table at the end.\\
Right now tsfm creates all the eps files for function logos, ID logos and KLD logos with the name of the Background Organism, so they will be overwritten if we create them in one run. So, that is why I generate them in order. it is not efficient! \\
\item[6. ] Later, I used the \textbf{bubbleV2.R} to generate the bubble plots one pdf file per functional class. This version of all.bubble.R script will take three path to skel file, tables, and output file and reads all the tables from folder site72, aligns them to sprinzl and creates the bubble plots.\\
All the outputs will be in directory /site72/Logos.\\
\end{itemize}
 
        
\renewcommand{\refname}{\spacedlowsmallcaps{References}} % For modifying the bibliography heading
\bibliographystyle{unsrt}
\bibliography{sample.bib} % The file containing the bibliography
\end{document}
