\documentclass[table,
12pt, % Main document font size
a4paper, % Paper type, use 'letterpaper' for US Letter paper
oneside, % One page layout (no page indentation)
%twoside, % Two page layout (page indentation for binding and different headers)
headinclude,footinclude, % Extra spacing for the header and footer
BCOR5mm, % Binding correction
]{scrartcl}
%%%%%%%%%%%%%%%%%%%%%%%%%%%%%%%%%%%%%%%%%
% Arsclassica Article
% Structure Specification File
%
% This file has been downloaded from:
% http://www.LaTeXTemplates.com
%
% Original author:
% Lorenzo Pantieri (http://www.lorenzopantieri.net) with extensive modifications by:
% Vel (vel@latextemplates.com)
%
% License:
% CC BY-NC-SA 3.0 (http://creativecommons.org/licenses/by-nc-sa/3.0/)
%
%%%%%%%%%%%%%%%%%%%%%%%%%%%%%%%%%%%%%%%%%

%----------------------------------------------------------------------------------------
%	REQUIRED PACKAGES
%----------------------------------------------------------------------------------------

\usepackage[
nochapters, % Turn off chapters since this is an article        
beramono, % Use the Bera Mono font for monospaced text (\texttt)
eulermath,% Use the Euler font for mathematics
pdfspacing, % Makes use of pdftex’ letter spacing capabilities via the microtype package
dottedtoc % Dotted lines leading to the page numbers in the table of contents
]{classicthesis} % The layout is based on the Classic Thesis style

\usepackage{arsclassica} % Modifies the Classic Thesis package

\usepackage[T1]{fontenc} % Use 8-bit encoding that has 256 glyphs

\usepackage[utf8]{inputenc} % Required for including letters with accents

\usepackage{graphicx} % Required for including images
\graphicspath{{Figures/}} % Set the default folder for images

\usepackage{enumitem} % Required for manipulating the whitespace between and within lists

\usepackage{lipsum} % Used for inserting dummy 'Lorem ipsum' text into the template

\usepackage{subfig} % Required for creating figures with multiple parts (subfigures)

\usepackage{amsmath,amssymb,amsthm} % For including math equations, theorems, symbols, etc

\usepackage{varioref} % More descriptive referencing

%----------------------------------------------------------------------------------------
%	THEOREM STYLES
%---------------------------------------------------------------------------------------

\theoremstyle{definition} % Define theorem styles here based on the definition style (used for definitions and examples)
\newtheorem{definition}{Definition}

\theoremstyle{plain} % Define theorem styles here based on the plain style (used for theorems, lemmas, propositions)
\newtheorem{theorem}{Theorem}

\theoremstyle{remark} % Define theorem styles here based on the remark style (used for remarks and notes)

%----------------------------------------------------------------------------------------
%	HYPERLINKS
%---------------------------------------------------------------------------------------

\hypersetup{
%draft, % Uncomment to remove all links (useful for printing in black and white)
colorlinks=true, breaklinks=true, bookmarks=true,bookmarksnumbered,
urlcolor=webbrown, linkcolor=RoyalBlue, citecolor=webgreen, % Link colors
pdftitle={}, % PDF title
pdfauthor={\textcopyright}, % PDF Author
pdfsubject={}, % PDF Subject
pdfkeywords={}, % PDF Keywords
pdfcreator={pdfLaTeX}, % PDF Creator
pdfproducer={LaTeX with hyperref and ClassicThesis} % PDF producer
}
\usepackage{xcolor}
\usepackage[a4paper, total={6.5in, 10in}]{geometry}
\usepackage[edges]{forest}
\usepackage{tikz-qtree}
\usepackage{tcolorbox}
%\usepackage[table, dvipsnames]{xcolor}
%\usetikzlibrary{shapes.geometric,arrows.meta}
\colorlet{shadecolor}{gray!10}
\usepackage{cite}
\usepackage{adjustbox}
\usepackage{booktabs}
\usepackage{float}
\hyphenation{Fortran hy-phen-ation} 
\captionsetup{font=footnotesize}
\usepackage{booktabs}
\usepackage{enumitem}
\usepackage{tabularx, makecell}%
\usepackage{tikz}
\usepackage{subfig}
\usepackage{graphicx}
\usepackage{cleveref}
\usetikzlibrary{shapes.geometric, arrows}
\renewcommand\theadfont{\normalsize\bfseries}
        \usepackage{etoolbox} %
        \AtBeginEnvironment{tabularx}{\setlist[enumerate, 1]{wide, leftmargin=*, itemsep=0pt, before=\vspace{-\dimexpr\baselineskip +2 \partopsep}, after=\vspace{-\baselineskip}}}
%----------------------------------------------------------------------------------------
%       TITLE AND AUTHOR(S)
%----------------------------------------------------------------------------------------
\title{\normalfont\spacedallcaps{}} % The article title
\begin{document}
%----------------------------------------------------------------------------------------
%       HEADERS
%----------------------------------------------------------------------------------------
\renewcommand{\sectionmark}[1]{\markright{\spacedlowsmallcaps{#1}}} % The header for all pages 
\lehead{\mbox{\llap{\small\thepage\kern1em\color{halfgray} \vline}\color{halfgray}\hspace{0.5em}\rightmark\hfil}} % The header style
\pagestyle{scrheadings} % Enable the headers specified in this block
\setcounter{tocdepth}{3} % Set the depth of the table of contents to show sections and subsections only
\newpage 
%Input: the foreground and background informations calculated by tsfm for each state. states are the set of alphabets used in input gene sequences which are {A,T,C,G,-}\\
\section{Background}
Review of the mathematics of function logos from [re?]:\\
A Structural feature is defined as a state $x \in X$ of a sequence at an alignment position l, $1 \leqslant l \leqslant L$, where L is equal to the length of the sequence, and $X = \{A, C, G, U, - \}$. The possible functional classes of tRNAs are denoted by $Y =$ \{A, C, D, E, F, G, H, I, J, K, L, M, N, P, Q, R, S, T, V, W, X, Y \}. Functional information $I_l(Y|x)$ that a state x confers about the frequencies of different classes $Y$ at position $l$ is calculated for each feature $x_l$ as  $I_l(Y|x)=H(Y)-e(n_l(x))-H_l(Y|x))$.\\
Function logo is a symbol-stacked-bar graph with positions on $x$ axis and informations on $y$ axis where each symbol within a bar is one of the functional classes of tRNAs in set $Y$ and symbols are sorted based on their height. Height of each symbol $y$ ($y \in Y$) for feature $x_l$ is proportional to the frequency of sequences of class $y$ with that feature and is calculated as $(\sum_{w \in Y} p_l(w|x)/p(w)) I_l(Y|x)$  [ref?].\\
\section{TSFM Method for calculating Information Difference and KullbackeLeibler divergence difference}
We used aligned tRNA sequences of Human and  each of the TriTryp classes of genomes as input. we divided the dataset into two alignments as foreground dataset $G^F$ and background dataset $G^B$. each of Homo and TriTryp are considered once as foreground and once as background for further analysis. We assumed that we have at least one sequence of each 21 tRNA classes annotated for each of the background and foreground dataset.\\
\subsection{Information Difference (ID) logos}  
ID logos \cite{FREYHULT20071276} visualize the evolutionary gain or loss of functional information between two genomes referred as foreground and background. Information difference of foreground vs background for a feature $X_l$ is calculated as $\Delta I(Y|X_l=x)=I^F(Y|X_l=x)-I^B(Y|X_l=x)$. In our information difference function logos the negative information difference is not displayed and the height of each symbol within a bar is proportional to $\frac{p^F(y|x_l)/p^F(y)}{P^B(y|x_l)/p^B(y)}$. $p^F(y|x_l)$ is the posterior probability of class membership given feature $X_l$; in other words, it is probability of a tRNA gene with feature $X_l$ being of class y. In this work, posterior probability of both background and foreground is calculated using pseudo counts when one of the 21 tRNA classes is absent for a feature $X_l$ in either background or foreground or both. pseudo counts are applied by adding one to the frequency of each of 21 classes. (note that based on our assumption prior will never becomes zero.)\\
\subsection{KullbackeLeibler divergence difference (KLD) logos}
KLD logos \cite{FREYHULT20071276} complement the Information Difference (ID) measure to visualize the changes in the functional associations of features of foreground genome against background genome. We calculate the height of KLD logos in each stack as $D_{KL}(Y|X_l=x)=D_{KL}(P^F(y|x_l)||p^B(y|x_l))= \sum_{y \in Y} p^F(y|x_l)log_2(\frac{P^F(y|x_l)}{p^B(y|x_l)})$ and height of each symbol within a bar proportional to $\frac{p^F(y|x_l)/p^F(y)}{P^B(y|x_l)/p^B(y)}$. The posterior probabilities are calculated with the same criteria described for ID logos. Further, we added two other restrictions in calculating the KL-divergence to overcome non-smoothness of this measure. First, We know that when $\lim_{p \to \infty}{p log p} = 0$. So, to avoid calculating logarithm of zero, whenever the foreground posterior probability for a feature $X_l$ becomes zero, we set the KLD to zero. Second, we know that when $p \neq 0$ and $\lim_{q \to 0}$, $kld(p||q)$ is defined as $\infty$. which means that when one event $e$ is possible in foreground distribution ($P^F(y|x_l) > 0$) and not possible in background distribution ($P^B(y|x_l) = 0$) then two distributions are absolutely different. However distributions are derived from observations and we must take into account the possibility of unseen events. In order to overcome this issue whenever the sample size of background having a feature $X_l$ is less than a small number (here we used 6), which will make the KLD a large number, we set the KLD to zero; in other words, we cannot say that two distributions are very different based on a small sample size so we won't display the KLD for this feature.\\
        
\renewcommand{\refname}{\spacedlowsmallcaps{References}} % For modifying the bibliography heading
\bibliographystyle{unsrt}
\bibliography{sample.bib} % The file containing the bibliography
\end{document}
